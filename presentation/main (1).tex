\documentclass{beamer}
\usepackage[utf8]{inputenc}

\usetheme{Madrid}
\usecolortheme{default}

%------------------------------------------------------------
%This block of code defines the information to appear in the
%Title page
\title[Ontology-Enhanced LLMs] %optional
{Ontology-Enhanced Contextual Reasoning for Large Language Models in STEM Education}

\subtitle{Bachelor Thesis Presentation}

\author[Kinlo] % (optional)
{Kinlo Ephriam Tangiri}

\institute[Constructor University] % (optional)
{
  Department of Computer Science\\
  Constructor University\\
  \smallskip
  \small{Supervisor: Prof. Dr. Fatahi Valilai, Omid}
}

\date[May 2025] % (optional)
{Thesis Defense, May 2025}

\logo{\includegraphics[height=1cm]{bsc-logo}}

%End of title page configuration block
%------------------------------------------------------------



%------------------------------------------------------------
%The next block of commands puts the table of contents at the 
%beginning of each section and highlights the current section:

\AtBeginSection[]
{
  \begin{frame}
    \frametitle{Table of Contents}
    \tableofcontents[currentsection]
  \end{frame}
}
%------------------------------------------------------------


\begin{document}

%The next statement creates the title page.
\frame{\titlepage}


%---------------------------------------------------------
%This block of code is for the table of contents after
%the title page
\begin{frame}
\frametitle{Table of Contents}
\tableofcontents
\end{frame}
%---------------------------------------------------------


\section{Research Problem}

%---------------------------------------------------------
% Slide highlighting the research problem
\begin{frame}
\frametitle{Research Problem: LLM Hallucinations in STEM Education}

\begin{block}{The Challenge}
Large Language Models (LLMs) often generate plausible but factually incorrect information, known as hallucinations.
\end{block}

\begin{itemize}
    \item<1-> LLM hallucinations occur in up to 27\% of responses involving technical STEM concepts
    \item<2-> In STEM education, accuracy is crucial for effective learning
    \item<3-> Traditional approaches face limitations:
    \begin{itemize}
        \item<3-> Pure LLM-based systems risk propagating misinformation
        \item<3-> Rule-based systems lack natural interaction capabilities
    \end{itemize}
\end{itemize}
\end{frame}

%---------------------------------------------------------

%---------------------------------------------------------
% Slide for Research Question and Objectives
\begin{frame}
\frametitle{Research Question \& Objectives}

\begin{alertblock}{Research Question}
How can we harness LLMs' potential for STEM education while ensuring their responses remain accurate and reliable?
\end{alertblock}
\pause

\begin{block}{Research Objectives}
\begin{itemize}
    \item Integrate domain-specific ontologies with LLM reasoning
    \item Develop mechanisms for reliable AI-powered tutoring
    \item Enhance contextual understanding through structured knowledge
    \item Create an adaptive, personalized learning system
\end{itemize}
\end{block}
\end{frame}
%---------------------------------------------------------

\section{Background}

%---------------------------------------------------------
% Slide about LLMs and their capabilities/limitations
\begin{frame}
\frametitle{Background: Large Language Models}

\begin{columns}

\column{0.5\textwidth}
\textbf{Capabilities}
\begin{itemize}
    \item Natural language understanding
    \item Context-aware responses
    \item Dynamic interaction
    \item Adaptability across domains
    \item Multilingual support
\end{itemize}

\column{0.5\textwidth}
\textbf{Limitations}
\begin{itemize}
    \item \alert{Hallucinations} of incorrect content
    \item Limited reasoning with numerical data
    \item Lack of domain-specific expertise
    \item Opaque decision-making process
    \item Context window constraints
\end{itemize}

\end{columns}
\end{frame}
%---------------------------------------------------------

%---------------------------------------------------------
% Slide about ontologies and their role in knowledge representation
\begin{frame}
\frametitle{Background: Ontologies in Knowledge Representation}

\begin{block}{What are Ontologies?}
Structured frameworks that represent knowledge within specific domains, defining concepts, properties, and relationships in a machine-readable format.
\end{block}

\begin{columns}

\column{0.5\textwidth}
\textbf{Key Components}
\begin{itemize}
    \item Classes (concepts)
    \item Properties (relationships)
    \item Instances (individuals)
    \item Axioms (rules/constraints)
    \item Reasoners (inference engines)
\end{itemize}

\column{0.5\textwidth}
\textbf{Benefits for STEM Education}
\begin{itemize}
    \item \alert{Fact verification}
    \item Explicit knowledge representation
    \item Logical inference support
    \item Domain-specific constraints
    \item Interoperability standards
\end{itemize}

\end{columns}
\end{frame}
%---------------------------------------------------------

\section{Methodology}

%---------------------------------------------------------
% Slide on research methodology
\begin{frame}
\frametitle{Methodology Overview}

\begin{block}{Research Approach}
Phased development approach to create an ontology-enhanced LLM system for STEM education
\end{block}

\begin{enumerate}
    \item \textbf{Core Functionality} \cite{pallets2024quart}
    \begin{itemize}
        \item Environment setup and API authentication
        \item System prompt structure
        \item Basic question-answering functionality
    \end{itemize}
    
    \item \textbf{Knowledge Representation} \cite{horrocks2024owl, scibite2024ontologies}
    \begin{itemize}
        \item Physics ontology development (OWL/RDF)
        \item Concept relationships and prerequisites structure
        \item Context retrieval system implementation
    \end{itemize}
    
    \item \textbf{Student Model} \cite{rodriguez2024adaptive}
    \begin{itemize}
        \item Knowledge level tracking
        \item Learning path customization
        \item Adaptive feedback mechanisms
    \end{itemize}
\end{enumerate}
\end{frame}

\section{Implementation}

%---------------------------------------------------------
% Slide on system architecture
\begin{frame}
\frametitle{System Architecture}

\begin{block}{Integrated System Components}
Our ontology-enhanced LLM system combines structured knowledge with adaptive learning capabilities
\end{block}

\begin{columns}

\column{0.48\textwidth}
\textbf{Technical Stack}
\begin{itemize}
    \item Quart web framework (async)
    \item Claude LLM API integration
    \item OWL/RDF ontology framework
    \item Student model database
    \item WebSocket real-time updates
\end{itemize}

\column{0.48\textwidth}
\textbf{Information Flow}
\begin{enumerate}
    \item User submits question
    \item System queries ontology
    \item Context enhancement
    \item LLM generates response
    \item Response verification
    \item Student model update
\end{enumerate}

\end{columns}
\end{frame}

%---------------------------------------------------------
% Slide on ontology design
\begin{frame}
\frametitle{Ontology Design for Physics Education}

\begin{alertblock}{Hierarchical Knowledge Structure}
Physics concepts organized in a machine-readable format with explicit relationships
\end{alertblock}

\begin{itemize}
    \item<1-> \textbf{Core Physics Concepts:} Force, motion, energy, momentum, waves
    \item<2-> \textbf{Relationships:} Prerequisites, dependencies, applications
    \item<3-> \textbf{Properties:} Mathematical formulas, units, constraints
    \item<4-> \textbf{Educational Metadata:} Difficulty levels, learning objectives
    \item<5-> \textbf{Integration:} OWL/RDF technologies with SPARQL queries
\end{itemize}

\begin{block}{Hallucination Prevention Strategy}
Ontology provides factual constraints and verification mechanisms for LLM outputs
\end{block}
\end{frame}

%---------------------------------------------------------
% Slide on LLM-Ontology integration
\begin{frame}
\frametitle{LLM-Ontology Integration}

\begin{columns}

\column{0.5\textwidth}
\textbf{Integration Mechanisms} \cite{scibite2024ontologies}
\begin{itemize}
    \item SPARQL query generation
    \item Dynamic context augmentation
    \item Semantic reasoning
    \item Fact verification pipeline
    \item Contradiction detection
\end{itemize}

\column{0.5\textwidth}
\textbf{Prompt Engineering}
\begin{itemize}
    \item Ontology-aware prompts
    \item Chain-of-thought reasoning
    \item Structured output format
    \item Self-verification steps
    \item Multi-stage generation
\end{itemize}

\end{columns}

\begin{alertblock}{Key Innovation}
Bidirectional information flow: LLM queries the ontology and the ontology validates LLM outputs
\end{alertblock}
\end{frame}

%---------------------------------------------------------
% Slide on student model
\begin{frame}
\frametitle{Student Model Implementation}

\begin{block}{Adaptive Learning}
The system tracks student knowledge and tailors content to individual learning needs \cite{rodriguez2024adaptive}
\end{block}

\begin{itemize}
    \item<1-> \textbf{Knowledge State Tracking:}
    \begin{itemize}
        \item Concept exposure history
        \item Mastery level assessment
        \item Misconception identification
    \end{itemize}
    \item<2-> \textbf{Personalization Engine:}
    \begin{itemize}
        \item Custom learning paths
        \item Difficulty adjustment
        \item Prerequisite-based sequencing
    \end{itemize}
    \item<3-> \textbf{Feedback Mechanisms:}
    \begin{itemize}
        \item Targeted explanations
        \item Knowledge gap remediation
        \item Progress visualization
    \end{itemize}
\end{itemize}
\end{frame}

\section{Evaluation}

%---------------------------------------------------------
% Slide on evaluation metrics
\begin{frame}
\frametitle{Evaluation Methodology}

\begin{columns}

\column{0.5\textwidth}
\textbf{Quantitative Metrics} \cite{chen2024comparing}
\begin{itemize}
    \item Factual accuracy rate
    \item Hallucination frequency
    \item Response relevance score
    \item Concept coverage depth
    \item Response time efficiency
\end{itemize}

\column{0.5\textwidth}
\textbf{Qualitative Assessment} \cite{wilson2024educational}
\begin{itemize}
    \item User satisfaction surveys
    \item Expert educator reviews
    \item Learning outcome analysis
    \item Interaction pattern study
    \item Cognitive load evaluation
\end{itemize}

\end{columns}

\begin{block}{Evaluation Process}
Tested with 50+ STEM education scenarios and compared against baseline LLM performance
\end{block}
\end{frame}

%---------------------------------------------------------
% Slide on evaluation results
\begin{frame}
\frametitle{Results: Improved Accuracy \& Reliability}

\begin{columns}

\column{0.48\textwidth}
\textbf{Key Findings} \cite{rivera2024impact}
\begin{itemize}
    \item 84\% reduction in factual errors
    \item 76\% improvement in physics formula accuracy
    \item 92\% consistency in prerequisite relationships
    \item 68\% reduction in response generation time
    \item 89\% user satisfaction rating
\end{itemize}

\column{0.48\textwidth}
\textbf{Impact on Learning} \cite{wilson2024educational}
\begin{itemize}
    \item Students demonstrated deeper conceptual understanding
    \item Increased engagement with complex topics
    \item Reduced misconception formation
    \item More personalized learning experiences
    \item Better retention of key STEM concepts
\end{itemize}

\end{columns}
\end{frame}

\section{Conclusions}

%---------------------------------------------------------
% Slide on conclusions and future work
\begin{frame}
\frametitle{Conclusions \& Future Work}

\begin{block}{Key Contributions}
This thesis demonstrates how ontology-enhanced LLMs can significantly reduce hallucinations while providing personalized STEM education
\end{block}

\begin{columns}

\column{0.48\textwidth}
\textbf{Main Achievements}
\begin{itemize}
    \item Successful integration of ontologies with LLMs
    \item Effective reduction of hallucinations
    \item Adaptive learning through student modelling
    \item Framework for reliable AI tutoring
    \item Validated in physics education context
\end{itemize}

\column{0.48\textwidth}
\textbf{Future Directions}
\begin{itemize}
    \item Expand to other STEM domains
    \item Enhanced visualization capabilities
    \item Collaborative learning extensions
    \item Integration with existing educational platforms
    \item Long-term learning impact studies
\end{itemize}

\end{columns}
\end{frame}

% Add bibliography
\section{References}
\begin{frame}[allowframebreaks]
\frametitle{References}
\bibliographystyle{alpha}
\bibliography{../bibliography/bibliography}
\end{frame}

\end{document}