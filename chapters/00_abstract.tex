% Abstract of the thesis

\section*{Abstract}

Large Language Models (LLMs), has transformed the way we interact with technology, 
yet its tendency to hallucinate, that is, to confidently output incorrect information presents a significant challenges, especially in educational applications 
where accuracy is very crucial.

This thesis aims to investigate whether integrating an ontology-driven knowledge models with LLMs can enhance their output for STEM education through more a 
reliable and less hallucinated contextually aware responses. To address the hallucination problem, we are proposing an approach that combines OWL and SPARQL technologies 
with domain-specific ontologies to systematically organize educational data like learner profiles, learning objectives, and instructional materials.

By integrating this structured knowledge as embeddings with Anthropic Claude's extensive context window and an open-source avatar frameworks, 
it will enable a more accurate and contextually-aware tutoring responses. The key benefits of this system include exponentially reducing hallucination, 
improved contextual understanding, and completely removing the need for prompt engineering, especially in early stages of education like secondary and high schools where students do not
know exactly what is relevant. This thesis hopes to demonstrate the significant advantages of an ontology-driven system over traditional LLM-only or rule-based approaches in creating personalized education responses.

\newpage 