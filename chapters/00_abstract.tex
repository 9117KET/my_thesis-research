% Abstract of the thesis

\section*{Abstract}

Large Language Models (LLMs) have transformed the way we interact with technology, 
yet their tendency to hallucinate (to confidently output incorrect information) presents significant challenges, especially in educational applications 
where accuracy is crucial.

This thesis aims to investigate whether integrating ontology-driven knowledge models with LLMs can enhance their output for STEM education through more 
reliable and contextually aware responses with reduced hallucinations (for the purpose of this research we will focus on Newton's laws in physics). To address the hallucination problem, we propose an approach that combines Ontology Web Language(OWL), and Resource Description Framework(RDF) and SPARQL technologies
to systematically organize educational data including learner profiles, learning objectives, and instructional materials.

Our methodology involves converting structured ontological knowledge into vector embeddings, which are then integrated with Anthropic Claude's LLM through its extensive context window. 
This integration, enables more accurate and contextually-aware tutoring interactions and LLM responses. We evaluate the system's effectiveness 
through comparative analysis of responses generated by our ontology-enhanced system versus our baseline model, with particular focus on factual accuracy and contextual relevance.

The benefits of this system include significantly reduced hallucination, improved contextual understanding, and elimination the need for prompt engineering requirements particularly in
secondary and high school education where students may not yet know what information is relevant. This thesis aims to demonstrate the advantages of an ontology-driven 
system over traditional LLM-only or rule-based approaches in creating personalized educational experiences.

\newpage