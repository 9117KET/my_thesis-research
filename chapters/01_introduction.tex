% Chapter 1: Introduction
\chapter{Introduction}
\label{chap:introduction}

% WHY - Opening with motivation and problem context
Large Language Models (LLMs) have transformed the way we use and apply artificial intelligence. 
Their ability to understand and generate human-like text offers numerous opportunities for educational applications. 
STEM education, in particular, could benefit from LLMs' capabilities to explain complex concepts, provide interactive tutoring, and adapt to individual learning needs especially
for schools that uses the one-size-fit all approach method for teaching.

However, LLMs face a critical challenge: hallucination. These models often generate plausible but factually incorrect or nonsensical information \cite{zhang2024survey}. 
In STEM education, where accuracy is extremely important, such hallucinations pose significant risks. Recent studies show that LLM hallucinations occur in up to 27\% of responses 
involving technical concepts \cite{huang2024survey}. This unreliability limits their deployment in educational settings.

% WHAT - Problem statement and proposed solution
This thesis addresses a fundamental question: How can we harness LLMs' potential for STEM education while ensuring their responses 
remain accurate and reliable? Traditional approaches fall short in two ways:
\begin{itemize}
    \item Pure LLM-based systems risk propagating misinformation through hallucinations \cite{su2024confabulation}
    \item Rule-based systems offer accuracy but lack the natural interaction capabilities needed for effective education
\end{itemize}

We propose an innovative solution: integrating ontological knowledge structures with LLM reasoning capabilities. This approach combines:
\begin{itemize}
    \item The structured precision of domain-specific ontologies \cite{nananukul2023halo}
    \item The natural language understanding of LLMs
    \item Real-time verification mechanisms \cite{hartl2024knowledge}
\end{itemize}

% HOW - Research objectives and approach
\section{Research Objectives}

Our research focuses on four primary objectives:

\begin{itemize}
    \item Develop a framework that integrates ontological knowledge with LLM reasoning
    \item Build an adaptive system that provides personalized, accurate learning experiences
    \item Create mechanism that reduces the rate of hallucinations using ontological constraints
\end{itemize}

\section{Research Contributions}

This work advances the field through several key contributions:

\begin{itemize}
    \item \textbf{Technical Innovation:}
        \begin{itemize}
            \item Ontology-enhanced LLM architecture for STEM education
            \item Efficient integration of structured and unstructured knowledge
        \end{itemize}
    
    \item \textbf{Educational Advancement:}
        \begin{itemize}
            \item Context-aware response generation for STEM concepts
            \item Adaptive learning pathways with accuracy guarantees
            \item Personalized feedback mechanisms
        \end{itemize}
    
    \item \textbf{Scientific Insights:}
        \begin{itemize}
            \item Methods to reduce hallucination in domain-specific LLM applications
            \item Techniques for semantic verification of LLM outputs
            \item Approaches to maintain engagement while ensuring accuracy
        \end{itemize}
\end{itemize}

\section{Technical Foundations}

Our approach builds upon recent advances in several areas:

\begin{itemize}
    \item \textbf{LLM Architecture:} Latest developments in transformer models and attention mechanisms \cite{zhang2024survey}
    \item \textbf{Ontology Engineering:} Semantic web technologies and knowledge representation \cite{doubletaken2024llm}
    \item \textbf{Hallucination Mitigation:} Recent techniques in fact verification and constraint satisfaction \cite{ji2023survey}
    \item \textbf{Educational Technology:} Adaptive learning systems and cognitive load theory
\end{itemize}

\section{Research Challenges}

We address two categories of challenges:

\subsection{Conceptual Challenges}
\begin{itemize}
    \item Bridging semantic gaps between ontologies and LLM representations
    \item Maintaining educational engagement while enforcing accuracy
    \item Developing metrics for hallucination detection and prevention
    \item Ensuring consistency across different knowledge domains
\end{itemize}

\subsection{Technical Challenges}
\begin{itemize}
    \item Efficient integration of OWL/SPARQL ontologies with LLM processing
    \item Real-time semantic verification of LLM outputs
    \item Scalable knowledge base management
    \item Optimization of response latency and resource usage
\end{itemize}

\subsection{Mitigation Strategies for Hallucination}
% This subsection introduces and compares the main strategies for mitigating hallucinations in LLMs, highlighting their challenges and the rationale for ontology-based approaches.

% WHY: Hallucination in LLMs is a critical barrier to their adoption in STEM education. Effective mitigation is essential for reliable, accurate, and trustworthy AI-driven learning systems.

Despite advances in LLM technology, hallucination remains a persistent problem. Researchers have explored three broad strategies to address this issue:

\begin{itemize}
    \item \textbf{Prompt Engineering:} This approach involves carefully designing input prompts to guide the model toward more accurate responses. For example, prompts can be enriched with context or specify the desired output format. While prompt engineering can reduce hallucinations in some cases, it requires significant manual effort and domain expertise. It is also difficult to scale and may not generalize well to new topics or user needs.
    \item \textbf{Fine-Tuning:} Fine-tuning retrains the model on curated datasets that emphasize accuracy and factual correctness. This can improve reliability, but maintaining up-to-date, high-quality datasets is resource-intensive. Continuous fine-tuning is often impractical for real-time educational settings due to cost and latency constraints.
    \item \textbf{Ontologies integration:} Integrating formal ontologies into LLM workflows provides structured, domain-specific knowledge. Ontologies define key concepts and relationships, enabling the model to reason with explicit, verifiable information. This approach supports retrieval-augmented generation (RAG) and real-time fact-checking, directly addressing hallucination by relating responses in trusted knowledge frameworks.
\end{itemize}

% WHAT: Each strategy has unique challenges that limit its effectiveness in STEM education.

Prompt engineering is limited by its manual nature and lack of scalability. Fine-tuning faces challenges in data curation and real-time applicability. Ontology-based grounding, while requiring initial investment in ontology development, offers a sustainable and scalable solution for ensuring accuracy and consistency, that is why our research focus on how impactful ontology models can reduce hallucinations in education applications.

% HOW: Why ontology-based grounding is best for STEM education.

Ontology-based approaches are particularly well-suited for STEM education because they:
\begin{itemize}
    \item Provide explicit definitions and relationships for complex concepts
    \item Enable automated, real-time verification of LLM outputs
    \item Support adaptive, context-aware learning experiences
    \item Facilitate cross-domain consistency and knowledge integration
\end{itemize}

By grounding LLM reasoning in ontological structures, we can systematically reduce hallucinations and deliver reliable, personalized educational content. This makes ontology-based mitigation the most promising strategy for high-stakes domains like STEM education.


\section{Thesis Structure}

The dissertation is organized as follows:

\begin{itemize}
    \item Chapter \ref{chap:background} examines the theoretical foundations and related work in ontology-enhanced LLMs
    \item Chapter \ref{chap:methodology} details our approach to integrating ontological knowledge with LLM reasoning
    \item Chapter \ref{chap:implementation} presents the system implementation and architecture
    \item Chapter \ref{chap:evaluation} provides empirical evaluation and results analysis
    \item Chapter \ref{chap:conclusion} discusses implications and future research directions
\end{itemize}

To facilitate further research, the complete source code for the ontology-enhanced LLM system developed in this thesis research process is available on GitHub at \url{https://github.com/9117KET/ai-avatar-ontology-integration-poc.git}. Additionally, a live demonstration of the system can be accessed at \url{https://ai-avatar-ontology-integration-poc.vercel.app/}. Further details about the repository structure and deployment are provided in Chapter \ref{chap:implementation}.